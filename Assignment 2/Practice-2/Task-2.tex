\documentclass{article}
\usepackage{color}
\title{\textbf{IMPACT OF DIGITALIZATION- AN ANALYSIS}}
\date{\textbf{2023-08-16}}
\author{\textbf{Tanvir Jahan Alin}}
\begin{document}
\maketitle
\tableofcontents
\pagebreak
\begin{abstract}
In this paper we discuss the purpose of digitalization to automate, add data quality, and collect and organize all that data
so that we can use advanced technology, such as better and smarter software but at the same time we have to deal with social and
ethical issues that arise as a result of digital integration. The Internet of Things, robots, biometrics, powerful technologies, real
reality and the unpopularity of taxpayers we see, and digital platforms. We are emphasizing the many advances in digital society
that seem to be rampant but at the same time we must conform to our social norms. This study shows that the new wave of digital
integration is putting pressure on these social norms. In order to successfully shape the digital community in a socially and ethical
way, participants need to have a clear understanding of what those problems might be. Management is greatly improved in areas of
privacy and data protection. In other ethical issues related to digital performance such as discrimination, independence, human
dignity and inequality of power, management is not well-organized. Social media is an interactive digital communication technology
that helps create or share / exchange information, ideas, career interests, and other ways to communicate with visible communities
and networks. 
\end{abstract}
\section{Introduction}
We have all been a part of Digital World where we have touched on Digitalized business processes through our day to date.
Think of activities like doing train bookings online, hotel bookings online, buying Air tickets, online bus tickets or paying with a
credit card, bank card, etc. For years, promoting Digitalization has been a Government initiative to provide all services to all citizens
on their web or electronic sites, making transactions transparent and smooth. The real changes due to Digitalisation are starting to
show today due to the push by govt., Which is expected to usher in a new era (such as how the computer performed in public and
in the private sector nearly two decades back). Things like paper money will soon run out. Digital integration brings major changes
in the lives of businesses and individuals. Digital technology influences the ways in which managers, employees and their clients
communicate both and communicate with relevant organizations. Many view Digital Humanities as a movement between traditional
humanity and the social sciences, which promises to bring digital technology to the cultural research questions. The same questions
that once required a lifetime of hand-crafted data processing a few weeks, or even a few days, with the help of digital data. Digital
Humanities should be seen as an extreme movement and as a discipline in itself right. Some are opposed to the official definition
of DH, seeing it as a small field and constantly changing, evading easy interpretation. to separate without the special field of
digital personality. Instead of trying to explain Digital Humanities.
\section {INTRODUCTION TO BUSINESS AND ECONOMY}
The escalation of digital technologies over the past years has been extraordinary. The number of Internet users, the number
of personal computers in use has grown rapidly. Digital transformation raising a frightful debate among policy-makers, industry
leaders and economists. Due to digitalization job losses globally will increased to 2 billion by 2030.The surge of ICT has not been
limited to only developed countries.
\subsection{Restricted economies}
Those with a digital integration rate of less than 25 -face challenges in accessing basic blocks of digital making such as wide
accessibility and affordability. In these nations, services remain expensive and limited access.
\subsection{Emerging economies}
Those with a score of between 25 and 30 - mostly they have faced the challenge of buying and gaining much.Expanded
digital production has impacted on a wide range of business ventures including corporate business models (BMs) by enabling a
variety of corporate partnerships leading to new product and service delivery and new forms of corporate and customer relationships.
At the same time, these digital innovations have put pressure on companies to think about their current strategy and to explore new
business opportunities systematically and initially. While digital research in the context of BMs is now receiving more attention,
the research gap still exists in this field because the amount of technical data is limited. This paper aims to discuss these issues.
\subsection{Design / method}
Appropriate artistic data collected from 12 key informants working in two different industries, the media and automotive
industries, were collected. Research is being conducted to examine the differences and similarities between how digital use affects
company value building, proposal and photography, and how firms deal with the challenges posed by digital growth.
\subsection{Findings}
The findings of the study show that, while digital performance is often considered important, the value proposition itself and
as a position in the value network determine the available options for a new model business (BMI) for digital performance. In
addition, organizational skills and personnel skills are identified as future challenges that will face both industries.
\subsection{Actual / value}
The findings of this study revealed that representatives of the media and automotive industry perceive the pressures and
opportunities for digital integration in relation to BMI; its use and exploitation, however, remains a challenge. This research
contributes to the body of existing knowledge by providing artistic information in a digital context and BMI.
\section{OBJECTIVES}
\subsection{ECONOMIC IMPACT}
Moreover, the economic impact of digital inclusion is as fast as in countries advancing to the most advanced stages. Limited
digital economies are understandable 0.5 per cent increase in GDP per capita on all 10 per cent increase in digital production,
while advanced digital economies accounted for 0.62 percent per capita GDP increase per 10\% increase in digital input.
\subsection{DIGITALIZATION}
Digital inclusion also has a huge impact on job creation as a whole economy: 10 percent increase in digital integration
reduces globalization the unemployment rate is 0.84 percent. From 2009 to 2010, digital usage has added an estimated 19 million
jobs to the global economy, from An estimated 18 million jobs were added from 2007 to 2008. This is especially so critical
detection of emerging markets, which will need to create hundreds of millions of jobs over the next ten years to ensure its
prosperity The number of young people can contribute to their country's economy
\subsection{ADVANCED PAYMENT SERVICE}
Unparalleled growth of the eCommerce market, fast, secure, and efficient payment options are essential. Therefore, an ecommerce platform needs to have an efficient and easy payment process. Although the developing economies are largely based
on cash, they are quick to adopt their digital capabilities.
\subsection{ENCOURAGES INNOVATION}
Digital transformation forces us to do something. And when the movement starts, the inertia will make it harder to stop.
The digital integration of a business often leads to innovation that allows you to become more aware of new styles and
opportunities offered by new technologies. In addition, it can also help to encourage the naming of team members (if they are
allowed to use it). Composing will depend not only on digital availability, but will also help achieve this.
\section{Conclusion}
Digital humanities are developing rapidly, both in terms of number of scholars and the means of engagement. Based on our
mapping of digital projects, we find that there is a breadth and depth of engagement across the humanities with digital
technologies. We are also in no doubt that digital technologies are creating the potential for conceptualizing radically new
research questions. The DH is facilitating new ways of research organization as evidenced by the crossover between humanities
and computer science. However, it is also clear that there are real challenges to the world of humanities as part of this
development. It is important not to fall into the trap of mental failure when it comes to exploring new threats. In the case of the
threat posed by technology used by terrorists, however, it seems that the participants in the debate fall into the trap of technology.
Technology has made our society to the point of not knowing the fact that there may be a day when our technology does not
work and at the moment, they cannot live without it. We have grown to rely on our technology to make our lives easier. If our
technology were to run out, our lives would be very difficult for a while until everyone learns to live without them. My solution
is to start that change soon.


\end{document}
